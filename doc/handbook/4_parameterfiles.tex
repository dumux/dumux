\section{Parameter Files in \Dumux}
\label{sec:inputFiles}
\Dumux simulations can be run with the use parameter files. Here basic information how to set,
extend, and improve your problem by using parameter files.
A list of all available parameters is provided in the \texttt{doxygen} documentation
of the file \texttt{parameterfile}, which is accessible via \texttt{Modules -> Parameters}.

\subsection{Advantages of Parameter Files}
Parameter files are worth of taking a closer look at, because using then considerably
improves the workflow.

\begin{itemize}
  \item The parameter file is read in by the compiled program. This way you can change
        values without having to recompile the whole application.
  \item With a very generic model, you can use different input files for defining different
        setups and always use the same program.
  \item You can use the parameter file in order to back up parameters that you used for
        a certain model run.
\end{itemize}

\subsection{Changing Parameters}
After having run the example application from section \ref{quick-start-guide} you will
get the following output at the end of the simulation run
\footnote{If you did not get the output, restart the application the following way:
\texttt{./test{\_}box2p -parameterFile ./test\_box2p.input -PrintParameters 1},
this will print the parameters once your simulation is finished}
:
\begin{lstlisting}[style=Bash]
# Run-time specified parameters:
[ Grid ]
File = "./grids/test_2p.dgf"
[ Implicit ]
EnableJacobianRecycling = "1"
EnablePartialReassemble = "1"
[ Problem ]
Name = "lensbox"
[ SpatialParams ]
LensLowerLeftX = "1.0"
LensLowerLeftY = "2.0"
LensUpperRightX = "4.0"
LensUpperRightY = "3.0"
[ TimeManager ]
DtInitial = "250"
TEnd = "3000"
# Compile-time specified parameters:
[ Implicit ]
EnableHints = "0"
MassUpwindWeight = "1"
MaxTimeStepDivisions = "10"
MobilityUpwindWeight = "1"
NumericDifferenceMethod = "1"
UseTwoPointFlux = "0"
[ LinearSolver ]
MaxIterations = "250"
PreconditionerRelaxation = "1"
ResidualReduction = "1e-06"
Verbosity = "0"
[ Newton ]
EnableResidualCriterion = "0"
EnableShiftCriterion = "1"
MaxRelativeShift = "1e-08"
MaxSteps = "18"
ResidualReduction = "1e-05"
SatisfyResidualAndShiftCriterion = "0"
TargetSteps = "10"
UseLineSearch = "0"
WriteConvergence = "0"
[ Problem ]
EnableGravity = "1"
[ TimeManager ]
MaxTimeStepSize = "1.79769e+308"
[ Vtk ]
AddVelocity = "0"
\end{lstlisting}

A number of things can be learned from this. Most prominently it tells you the parameters,
that can easily be added to the input file without having to change anything in the source code.
The output will tell you, which parameters are available to the problem and whether they have
been specified
\begin{itemize}
  \item \emph{run-time} via your input file
  \item \emph{compile-time} and have not been overwritten by the input file
  \item in your input file, but are \emph{UNUSED} by the simulation
\end{itemize}
For example by adding
\begin{lstlisting}[style=Bash]
[ Newton ]
MaxRelativeShift = "1e-11"
\end{lstlisting}
to the input file you can specify that the Newton solver considers itself converged for an
error a thousand times smaller.

The \emph{UNUSED} warning
\begin{lstlisting}[style=Bash]
# UNUSED parameters:
Problem.ImportantVariable = "42"
\end{lstlisting}
is important, because it shows that the application did not read in this value.
Maybe because it was attributed to the wrong group or there was a typo.
This feature is \emph{very} useful for debugging or spotting typos, like when you wanted
to overwrite one of the parameters listed under \texttt{Compile-time specified parameters}
and misspelled it in the input file, it will be listed in the \texttt{UNUSED parameters} section.

\subsection{Technical Issues on Parameters}
In case you want to learn more about how the input files work, please have a
look at the very helpful \Dune documentation, look for
\texttt{Dune::ParameterTree}.

The parameter tree can also be filled without the help of a text file.
Everything that is specified in a \Dumux input file can also be specified
directly on the command line. If there is also an input file, the respective
parameter on the command line has precedence.

All applications have a help message which you can read by giving
\texttt{--help}   as a command line argument to the application. A message
listing syntax and the mandatory input will be displayed on the command line.


