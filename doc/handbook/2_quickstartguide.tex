\section[Quick Start Guide]{Quick Start Guide: The First Run of a Test Application}
\label{quick-start-guide}

The previous section showed how to install and compile \Dumux. This chapter
shall give a very brief introduction how to run a first test application and how
to visualize the first output files. A more detailed explanations can be found in
the tutorials in the following chapter.\\
All executables are compiled in the \texttt{build} subdirectories of \Dumux.
If not given differently in the input files, this is \texttt{build-cmake} as default.

\begin{enumerate}
\item Go to the directory \texttt{build-cmake/test}. There, various test application
      folders can be found. Let us consider as example \texttt{implicit/test{\_}box2p}:
\item Enter the folder \texttt{implicit/2p}. Type \texttt{make test{\_}box2p} in order
      to compile the application \texttt{test{\_}box2p}. To run the simulation, type\\
      \texttt{./test{\_}box2p -parameterFile ./test\_box2p.input}\\
      into the console. The parameter \texttt{-parameterFile} specifies that all
      important parameters (like first timestep size, end of simulation and location
      of the grid file) can be found in a text file in the same directory  with the
      name \texttt{test\_box2p.input}.
\item The simulation starts and produces some .vtu output files and also a .pvd
      file. The .pvd file can be used to examine time series and summarizes the .vtu
      files. It is possible to stop a running application by pressing $<$Ctrl$><$c$>$.
\item You can display the results using the visualization tool ParaView (or
      alternatively VisIt). Just type \texttt{paraview} in the console and open the
      .pvd file. On the left hand side, you can choose the desired parameter to be displayed.
\end{enumerate}
