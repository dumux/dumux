%%%%%%%%%%%%%%%%%%%%%%%%%%%%%%%%%%%%%%%%%%%%%%%%%%%%%%%%%%%%%%%%%
% This file has been autogenerated from the LaTeX part of the   %
% doxygen documentation; DO NOT EDIT IT! Change the model's .hh %
% file instead!!                                                %
%%%%%%%%%%%%%%%%%%%%%%%%%%%%%%%%%%%%%%%%%%%%%%%%%%%%%%%%%%%%%%%%%

This model implements a one-\/phase flow of an incompressible fluid, that consists of two components. The deformation of the solid matrix is described with a quasi-\/stationary momentum balance equation. The influence of the pore fluid is accounted for through the effective stress concept (Biot 1941). The total stress acting on a rock is partially supported by the rock matrix and partially supported by the pore fluid. The effective stress represents the share of the total stress which is supported by the solid rock matrix and can be determined as a function of the strain according to Hooke\textquotesingle{}s law.

As an equation for the conservation of momentum within the fluid phase Darcy\textquotesingle{}s approach is used\-: \[ v = - \frac{\textbf K}{\mu} \left(\textbf{grad}\, p - \varrho_w {\textbf g} \right) \]

Gravity can be enabled or disabled via the property system. By inserting this into the volume balance of the solid-\/fluid mixture, one gets \[ \frac{\partial \text{div} \textbf{u}}{\partial t} - \text{div} \left\{ \frac{\textbf K}{\mu} \left(\textbf{grad}\, p - \varrho_w {\textbf g} \right)\right\} = q \;, \]

The transport of the components $\kappa \in \{ w, a \}$ is described by the following equation\-: \[ \frac{ \partial \phi_{eff} X^\kappa}{\partial t} - \text{div} \left\lbrace X^\kappa \frac{{\textbf K}}{\mu} \left( \textbf{grad}\, p - \varrho_w {\textbf g} \right) + D^\kappa_\text{pm} \frac{M^\kappa}{M_\alpha} \textbf{grad} x^\kappa - \phi_{eff} X^\kappa \frac{\partial \boldsymbol{u}}{\partial t} \right\rbrace = q. \]

If the model encounters stability problems, a stabilization term can be switched on. The stabilization term is defined in Aguilar et al (2008)\-: \[ \beta \text{div} \textbf{grad} \frac{\partial p}{\partial t} \] with $\beta$\-: \[ \beta = h^2 / 4(\lambda + 2 \mu) \] where $h$ is the discretization length.

The balance equations with the stabilization term are given below\-: \[ \frac{\partial \text{div} \textbf{u}}{\partial t} - \text{div} \left\{ \frac{\textbf K}{\mu} \left(\textbf{grad}\, p - \varrho_w {\textbf g} \right) + \varrho_w \beta \textbf{grad} \frac{\partial p}{\partial t} \right\} = q \;, \]

The transport of the components $\kappa \in \{ w, a \}$ is described by the following equation\-:

\[ \frac{ \partial \phi_{eff} X^\kappa}{\partial t} - \text{div} \left\lbrace X^\kappa \frac{{\textbf K}}{\mu} \left( \textbf{grad}\, p - \varrho_w {\textbf g} \right) + \varrho_w X^\kappa \beta \textbf{grad} \frac{\partial p}{\partial t} + D^\kappa_\text{pm} \frac{M^\kappa}{M_\alpha} \textbf{grad} x^\kappa - \phi_{eff} X^\kappa \frac{\partial \boldsymbol{u}}{\partial t} \right\rbrace = q. \]

The quasi-\/stationary momentum balance equation is\-: \[ \text{div}\left( \boldsymbol{\sigma'}- p \boldsymbol{I} \right) + \left( \phi_{eff} \varrho_w + (1 - \phi_{eff}) * \varrho_s \right) {\textbf g} = 0 \;, \] with the effective stress\-: \[ \boldsymbol{\sigma'} = 2\,G\,\boldsymbol{\epsilon} + \lambda \,\text{tr} (\boldsymbol{\epsilon}) \, \boldsymbol{I}. \]

and the strain tensor $\boldsymbol{\epsilon}$ as a function of the solid displacement gradient $\textbf{grad} \boldsymbol{u}$\-: \[ \boldsymbol{\epsilon} = \frac{1}{2} \, (\textbf{grad} \boldsymbol{u} + \textbf{grad}^T \boldsymbol{u}). \]

Here, the rock mechanics sign convention is switch off which means compressive stresses are $<$ 0 and tensile stresses are $>$ 0. The rock mechanics sign convention can be switched on for the vtk output via the property system.

The effective porosity is calculated as a function of the solid displacement\-: \[ \phi_{eff} = \frac{\phi_{init} + \text{div} \boldsymbol{u}}{1 + \text{div}} \] All equations are discretized using a vertex-\/centered finite volume (box) or cell-\/centered finite volume scheme as spatial and the implicit Euler method as time discretization.

The primary variables are the pressure $p$ and the mole or mass fraction of dissolved component $x$ and the solid displacement vector $\boldsymbol{u}$.

