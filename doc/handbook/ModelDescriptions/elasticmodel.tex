%%%%%%%%%%%%%%%%%%%%%%%%%%%%%%%%%%%%%%%%%%%%%%%%%%%%%%%%%%%%%%%%%
% This file has been autogenerated from the LaTeX part of the   %
% doxygen documentation; DO NOT EDIT IT! Change the model's .hh %
% file instead!!                                                %
%%%%%%%%%%%%%%%%%%%%%%%%%%%%%%%%%%%%%%%%%%%%%%%%%%%%%%%%%%%%%%%%%

This model implements a linear elastic solid using Hooke\textquotesingle{}s law as stress-\/strain relation and a quasi-\/stationary momentum balance equation\-: \[ \boldsymbol{\sigma} = 2\,G\,\boldsymbol{\epsilon} + \lambda \,\text{tr} (\boldsymbol{\epsilon}) \, \boldsymbol{I}. \]

with the strain tensor $\boldsymbol{\epsilon}$ as a function of the solid displacement gradient $\textbf{grad} \boldsymbol{u}$\-: \[ \boldsymbol{\epsilon} = \frac{1}{2} \, (\textbf{grad} \boldsymbol{u} + \textbf{grad}^T \boldsymbol{u}). \]

Gravity can be enabled or disabled via the property system. By inserting this into the momentum balance equation, one gets \[ \text{div} \boldsymbol{\sigma} + \varrho {\textbf g} = 0 \;, \]

The equation is discretized using a vertex-\/centered finite volume (box) scheme as spatial discretization.

