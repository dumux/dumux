% SPDX-FileCopyrightInfo: Copyright © DuMux Project contributors, see AUTHORS.md in root folder
% SPDX-License-Identifier: CC-BY-4.0

\section{Restart \Dumux Simulations}
\label{sc_restartsimulations}

\Dumux has some experimental support for check-pointing (restarting paused/stopped/crashed simulations).
You can restart a \Dumux simulation from any time point where a VTK file was written out.
This is currently only supported for sequential, non-adaptive simulations. For adaptive simulation
the full hierarchical grid has to be stored. This is usually done with the grid's \texttt{BackupRestoreFacility}.
There is currently no special support by \Dumux for that, but it is possible to implement
a restart using \texttt{BackupRestoreFacility} with plain Dune.

For VTK files the output can be read with the free function \texttt{loadSolution}. Grids can be read with
the \texttt{Dumux::VTKReader} or you can simply recreate the grid as you did in the first simulation run.

Writing double-precision floating point numbers to VTK files is available since \Dune release 2.7. If you are using that version, it is now possible to specify output precision in the input file using \texttt{Vtk.Precision} followed by either \texttt{Float32}, \texttt{Float64}, \texttt{UInt32}, \texttt{UInt8} or \texttt{Int32}. \texttt{Float32} is set as the default. We especially advice the use of \texttt{Float64} when working with restart files.

The restart capabilities will hopefully be improved in future versions of \Dumux-3.
We are looking forward to any contributions (especially HDF5 / XDMF support, improvement of VTK support).
