\section{Developing \Dumux}
\label{sc_developingdumux}

\subsection{Communicate with \Dumux Developers}

\paragraph{Issues and Bug Tracking}
The bug-tracking system \emph{GitLab Issues} offers the possibility to report bugs or discuss new development requests.
Feel free to register (if you don't have an account at out \emph{Git} yet) and to contribute
at \url{https://git.iws.uni-stuttgart.de/dumux-repositories/dumux/issues}.

\paragraph{Commits, Merges, etc.}
To be up-to-date with the latest changes made to any git-repository, you can use RSS Feeds.
Simply click on \emph{Issues} or \emph{Activity} and then select a tab you are interested in
and use your favorite RSS-application for receiving the news.

\paragraph{Automatic Testing Dashboard}
The automatic testing using \emph{BuildBot} helps to constantly check the
\Dumux problems for compiling and running correctly. It is available at
\url{https://git.iws.uni-stuttgart.de/buildbot/#/builders}.

\paragraph{The General Mailing List:}
If you have questions, specific problems (which you really struggle to solve on your own),
or hints for the \Dumux-developers, please contact the mailing list \url{dumux@iws.uni-stuttgart.de}.
You can subscribe to the mailing list via
\url{https://listserv.uni-stuttgart.de/mailman/listinfo/dumux}, then you
will be informed about upcoming releases or events.

\subsection{Coding Guidelines}
Writing code in a readable manner is very important, especially
for future code developers (e.g. for adding features, debugging, etc.).
For the style guide and instructions how to contribute to \Dumux visit
\url{https://git.iws.uni-stuttgart.de/dumux-repositories/dumux/blob/master/CONTRIBUTING.md}.


\subsection{Tips and Tricks}
\Dumux users and developers at the LH2 are also referred to the internal confluence pages for
more information.

\paragraph{Optimized computation vs debugging}
\Dune and \Dumux are built with the help of \texttt{dunecontrol}.
Per default, \Dumux is compiled using optimization options, which leads to faster runtimes but is unsuitable
for debugging. For debug opts you can set \texttt{DCMAKE\textunderscore BUILD\textunderscore TYPE} to \texttt{Debug} or \texttt{RelWithDebInfo}
in your options file. You can also do this in any of the \texttt{CMakeLists.txt} in Dumux by adding:

\begin{lstlisting}[style=Shell]
set(CMAKE_BUILD_TYPE Debug)
\end{lstlisting}

Afterwards rerun cmake (run cmake \texttt{$<$path-to-build-dir$>$}).

\paragraph{Dunecontrol for selected modules}
A complete build using \texttt{dunecontrol} takes some time. In many cases not all modules need to be re-built.
Pass the flag \texttt{--only=dumux} to \texttt{dunecontrol} for configuring or building only \Dumux. A more
complex example would be a case in which you have to configure and build only e.g. \Dune{}-grid
and \Dumux. This is achieved by adding \texttt{--only=MODULE,dumux}.

\paragraph{Patching Files or Modules}
If you want to send changes to an other developer of \Dumux providing patches
can be quite smart. To create a patch simply type:
\begin{lstlisting}[style=Bash]
$ git diff > PATCHFILE
\end{lstlisting}
\noindent which creates a text file containing all your changes to the files
in the current folder or its subdirectories.
To apply a patch in the same directory type:
\begin{lstlisting}[style=Bash]
$ patch -p1 < PATCHFILE
\end{lstlisting}

%TODO: currently, no DUNE patches necessary! Thus, this section is commented and the missing reference would be bad.
% Uncomment the following statement again when patches might be necessary.
% See \ref{sc:patchingDUNE} if you need to apply patches to \Dumux or \Dune.

\paragraph{File Name and Line Number by Predefined Macro}
If you want to create output in order to later know where some output or debug information came from, use the predefined
macros \texttt{\_\_FILE\_\_} and \texttt{\_\_LINE\_\_}:
\begin{lstlisting}[style=DumuxCode]
std::cout << "# This was written from "<< __FILE__ << ", line " << __LINE__ << std::endl;
\end{lstlisting}

\paragraph{Using \Dune Debug Streams}
\Dune provides a helpful feature for keeping your debug-output organized.
It uses simple streams like \texttt{std::cout}, but they can be switched on and off
for the whole project. You can choose five different levels of severity:
\begin{verbatim}
5 - grave (dgrave)
4 - warning (dwarn)
3 - info (dinfo)
2 - verbose (dverb)
1 - very verbose (dvverb)
\end{verbatim}
\noindent They are used as follows:
\begin{lstlisting}[style=DumuxCode]
// define the minimal debug level somewhere in your code
#define DUNE_MINIMAL_DEBUG_LEVEL 4
Dune::dgrave << "message"; // will be printed
Dune::dwarn << "message"; // will be printed
Dune::dinfo << "message"; // will NOT be printed
\end{lstlisting}

\paragraph{Make headercheck:}
To check one header file for all necessary includes to compile the contained code, use \texttt{make headercheck}.
Include the option \texttt{-DENABLE\_HEADERCHECK=1} in your opts file and run \texttt{dunecontrol}.
Then go to the top level in your build-directory and type \texttt{make headercheck} to check all headers
or press 'tab' to use the auto-completion to search for a specific header.
