% SPDX-FileCopyrightInfo: Copyright © DuMux Project contributors, see AUTHORS.md in root folder
% SPDX-License-Identifier: CC-BY-4.0

\section{Steps of a \Dumux Simulation}
\label{flow}


This chapter is supposed to give a short overview over how things are ``handed around'' in \Dumux. It
is not a comprehenisve guide through the modeling framework of \Dumux, but
hopefully it will help getting to grips with it.

In Section \ref{content} the structure of \Dumux is shown from a \emph{content}
point of view.

\subsection{Structure -- by Content}

\label{content}
In Figure \ref{fig:algorithm}, the algorithmic representations of a monolithical
solution scheme is illustrated down to the element level.

\begin{figure}[hbt]
\setcounter{thingCounter}{0}

\scriptsize
\sffamily
\begin{center}\parbox{0cm}{
\begin{tabbing}
\textbf{{\begin{turn}{45}\color{black}\numberThis{main}{init}\end{turn}}}             \=
\textbf{{\begin{turn}{45}\color{dumuxBlue}\numberThis{time step}{prep}\end{turn}}}            \=
\textbf{{\begin{turn}{45}\color{Mulberry}\numberThis{\textsc{Newton}}{elem}\end{turn}}}         \=
\textbf{{\begin{turn}{45}\color{dumuxYellow}\numberThis{element}{calc}\end{turn}}}             \=  \\
\\
\color{black}initialize \\
\color{black}\textbf{foreach} time step\\

  \> \color{dumuxBlue}\textbf{foreach} \textsc{Newton} iteration \\

    \> \> \color{Mulberry}\textbf{foreach} element \\

      \> \> \> \color{dumuxYellow}- calculate element \\
      \> \> \> \color{dumuxYellow}\; residual vector and \\
      \> \> \> \color{dumuxYellow}\; Jacobian matrix\\
      \> \> \> \color{dumuxYellow}- assemble into global\\
      \> \> \> \color{dumuxYellow}\; residual vector and \\
      \> \> \> \color{dumuxYellow}\;{Jacobian} matrix \\

    \> \> \color{Mulberry}\textbf{endfor} \\

    \> \> \color{Mulberry}solve linear system\\
    \> \> \color{Mulberry}update solution\\
    \> \> \color{Mulberry}check for \textsc{Newton} convergence\\
  \> \color{dumuxBlue}\textbf{endfor}\\
  \> \color{dumuxBlue}- adapt time step size, \\
  \> \color{dumuxBlue}\; possibly redo with smaller step size\\
  \> \color{dumuxBlue}- write result\\
\color{black}\textbf{endfor}\\
\color{black}finalize
\end{tabbing}}
\end{center}
\caption{Structure of a monolithical solution scheme in \Dumux.}
\label{fig:algorithm}
\end{figure}

\subsection{Structure -- by Implementation}
A possible starting point to understand how the above mentioned algorithm is implemented within \Dumux,
is the example main file
\url{https://git.iws.uni-stuttgart.de/dumux-repositories/dumux-course/-/blob/releases/\DumuxVersion/exercises/exercise-mainfile/exercise1pamain.cc}
