\section{Restart \Dumux Simulations}
\label{sec:restartSimulations}

Using the restart capability of \Dumux can be advantageous for computationally
expensive or time consuming simulations, because you can restart the simulation
from a specific point in time and e.g. extend the simulation beyond the originally
end of simulation. What you need is a \texttt{*.drs} file (which contains the
all necessary restart information.
Then you can simply restart a simulation via
\begin{lstlisting}[style=Bash]
./test_program -ParameterFile test_program.input -TimeManager.Restart RESTART_TIME
\end{lstlisting}
To the test restart behavior e.g. use the \texttt{test\_box1p2cni} problem
in the \texttt{test/implicit/1p2c} folder.
You get the \texttt{RESTART\_TIME} from the name of your \texttt{.drs} file.
Please note, that restarting will only work by giving a exact time from
an existing restart file.
Depending on your type of model, you should get a \texttt{.drs} file every
5th or 10th time step. If this not frequently enough, you can change it
by using the following function into your problem header:
\begin{lstlisting}[style=DumuxCode]
/*!
 * \brief Returns true if a restart file should be written to
 *        disk.
 */
bool shouldWriteRestartFile() const
{
  return true;
}
\end{lstlisting}

