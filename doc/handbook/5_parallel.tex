\section{Parallel Computation}
\label{sec:parallelcomputation}
Multicore processors are standard nowadays and parallel programming is the key to gain
performance from modern computers. This section explains how \Dumux can be used 
on multicore systems, ranging from the users desktop computer to high performance
computing clusters.  

There are different concepts and methods for parallel programming, which are
often grouped in \textit{shared-memory} and \textit{distributed-memory}  
approaches. The parallelization in \Dumux is based on the 
\textit{Message Passing Interface} (MPI), which is usually called MPI parallelization (distributed-memory approach). 
It is the MPI parallelization that allows the user to run
\Dumux applications in parallel on a desktop computer, the users laptop or 
large high performance clusters. However, the chosen \Dumux 
model must support parallel computations, which is the case for the most \Dumux applications.

The main idea behind the MPI parallelization is the concept of \textit{domain 
decomposition}. For parallel simulations, the computational domain is split into 
subdomains and one process (\textit{rank}) is used to solves the local problem of each 
subdomain. During the global solution process, some data exchange between the 
ranks/subdomains is needed. MPI is used to send data to other ranks and to receive 
data from other ranks. 
Most grid managers contain own domain decomposition methods to split the 
computational domain  into subdomains. Some grid managers also support external 
tools like METIS, ParMETIS, PTScotch or ZOLTAN for partitioning.

Before \Dumux can be started in parallel, a 
MPI library (e.g. OpenMPI, MPICH or IntelMPI) 
must be installed on the system and all \Dune modules and \Dumux must be recompiled.  


\subsection{Prepare an Parallel Application}
Not all parts of \Dumux can be used in parallel. One example are the linear solvers
of the sequential backend. However, with the AMG backend \Dumux provides 
a parallel solver backend based on Algebraic Multi Grid (AMG) that can be used in
parallel. 
If an application uses not already the AMG backend, the 
user must switch the backend to AMG to run the application also in parallel.

First, the header file for the parallel AMG backend must be included.

\begin{lstlisting}[style=DumuxCode]
#include <dumux/linear/amgbackend.hh>
\end{lstlisting}

so that the backend can be used. The header file of the sequential backend

\begin{lstlisting}[style=DumuxCode]
#include <dumux/linear/seqsolverbackend.hh>
\end{lstlisting}
can be removed.

Second, the linear solver must be switched to the AMG backend 

\begin{lstlisting}[style=DumuxCode]
using LinearSolver = Dumux::AMGBackend<TypeTag>;
\end{lstlisting}

and the application must be compiled. 

\subsection{Run an Parallel Application}
The starting procedure for parallel simulations depends on the chosen MPI library. 
Most MPI implementations use the \textbf{mpirun} command

\begin{lstlisting}[style=Bash]
mpirun -np <n_cores> <executable_name>
\end{lstlisting}

where \textit{-np} sets the number of cores (\texttt{n\_cores}) that should be used for the 
computation. On a cluster you usually have to use a queuing system (e.g. slurm) to 
submit a job. 

\subsection{Handling Parallel Results}
For most models, the results should not differ between parallel and serial 
runs. However, parallel computations are not naturally deterministic. 
A typical case where one can not assume a deterministic behaviour are models where
small differences in the solution can cause large differences in the results 
(e.g. for some turbulent flow problems). Nevertheless, it is useful to expect that
the simulation results do not depend on the number of cores. Therefore you should double check 
the model, if it is really not deterministic. Typical reasons for a wrong non deterministic
behaviour are errors in the parallel computation of boundary conditions or missing/reduced
data exchange in higher order gradient approximations. Also you should keep in mind, that 
for iterative solvers there can occur differences in the solution due to the error threshold.


For serial computations \Dumux produces single vtu-files as default output format. 
During a simulation, one vtu-file is written for every output step. 
In the parallel case, one vtu-file for each step and processor is created. 
For parallel computations an additional variable "process rank" is written 
into the file. The process rank allows the user to inspect the subdomains 
after the computation.

\subsection{MPI scaling}
For parallel computations the number of cores must be chosen 
carefully. Using too many cores will not always lead to more performance, but 
can produce a bad efficiency. One reason is that for small subdomains, the 
communication between the subdomains gets the limiting factor for parallel computations. 
The user should test the MPI scaling (relation between the number of cores and the computation time) 
for each specific application to ensure a fast and efficient use of the given resources.   
